\documentclass[a4paper,skipsamekey,10pt,english]{curve}

% Default biblatex style used for the publication list is APA6. If you wish to use a different style or pass other options to biblatex you can change them here. 
\PassOptionsToPackage{style=ieee,sorting=ydnt,uniquename=init,defernumbers=true}{biblatex}

% Most commands and style definitions are in settings.sty.
\usepackage{settings}

% If you need to further customise your biblatex setup e.g. with \DeclareFieldFormat etc please add them here AFTER loading settings.sty. For example, to remove the default "[Online] Available:" prefix before URLs when using the IEEE style:
\DefineBibliographyStrings{english}{url={\textsc{url}}}

%% Only needed if you want a Publication List
\addbibresource{own-bib.bib}

%% Specify your last name(s) and first name(s) (as given in the .bib) to automatically bold your own name in the publications list. 
%% One caveat: You need to write \bibnamedelima where there's a space in your name for this to work properly; or write \bibnamedelimi if you use initials in the .bib
% \mynames{Steven\bibnamedelima Hicks}

%% You can specify multiple names like this, especially if you have changed your name or if you need to highlight multiple authors. See items 6–9 in the example "Journal Articles" output.
\mynames{Steven\bibnamedelima Hicks,
  Steven\bibnamedelima Alexander\bibnamedelima Hicks,
  Steven\bibnamedelima A\bibnamedelima Hicks}
%% MAKE SURE THERE IS NO SPACE AFTER THE FINAL NAME IN YOUR \mynames LIST

% Change the fonts if you want
\ifxetexorluatex % If you're using XeLaTeX or LuaLaTeX
  \usepackage{fontspec} 
  %% You can use \setmainfont etc; I'm just using these font packages here because they provide OpenType fonts for use by XeLaTeX/LuaLaTeX anyway
  \usepackage[p,osf,swashQ]{cochineal}
  \usepackage[medium,bold]{cabin}
  \usepackage[varqu,varl,scale=0.9]{zi4}
\else % If you're using pdfLaTeX or latex
  \usepackage[T1]{fontenc}
  \usepackage[p,osf,swashQ]{cochineal}
  \usepackage{cabin}
  \usepackage[varqu,varl,scale=0.9]{zi4}
\fi

\renewcommand{\arraystretch}{1}

% Change the page margins if you want
\newcommand{\marginlengths}{.5cm}
\geometry{left=\marginlengths,right=\marginlengths,top=\marginlengths,bottom=\marginlengths}

% Change the colours if you want
% \definecolor{SwishLineColour}{HTML}{00FFFF}
% \definecolor{MarkerColour}{HTML}{0000CC}

% Change the item prefix marker if you want
% \prefixmarker{$\diamond$}

%% Photo is only shown if "fullonly" is included
\includecomment{fullonly}
% \excludecomment{fullonly}


%%%%%%%%%%%%%%%%%%%%%%%%%%%%%%%%%%%%%%


\leftheader{%
  {\LARGE\bfseries\sffamily Steven Hicks, Ph.D.}
  
  \makefield{\faEnvelope[regular]}{\href{mailto:steven@kefka.xyz}{\texttt{steven@kefka.xyz}}}
  \makefield{\faGlobe}{\url{https://stevenhicks.xyz}}
  \makefield{\faGoogle}{\href{https://scholar.google.com/citations?user=2fVVFSwAAAAJ&hl=en}{\texttt{Google Scholar}}}
  \makefield{\faLinkedin}
  {\href{https://www.linkedin.com/in/steven-hicks-1588a014b/}{\texttt{LinkedIn}}}
  \makefield{\faGithub}{\href{https://github.com/stevenah}{\texttt{stevenah}}}
}

\rightheader{~}
\begin{fullonly}
% \photo[r]{photo}
% \photoscale{0.2}
\end{fullonly}

\title{Curriculum Vitae}

\begin{document}
\makeheaders[c]

\begin{rubric}{Bio}
\entry*[] Steven Hicks is a researcher at SimulaMet and has a strong background in computer science and AI. He specializes in the interdisciplinary application of AI methodologies across various domains and modalities. A central part of his research is the transparent and explainable use of AI, with an emphasis on ethical and reliable solutions. His research involves the use of cutting-edge AI technologies to solve complex problems in sectors such as healthcare, sports, education, and telecommunications. In the project team, he brings deep technological expertise in AI, covering both the underlying technology and various application areas.
\end{rubric}
%!TEX encoding = UTF8
%!TEX root =cv-llt.tex

\begin{rubric}{Skills}
\entry*[Programming Languages]
    Proficient in Java, \smallcaps{SQL}, Bash, C, C++, C\#, .Net, Python, and JavaScript. Experience with Julia, Rust, and Go.
\entry*[Databases\hfill]
    Skilled in database design and management using SQL, MySQL, PostgreSQL, and familiarity with NoSQL databases like MongoDB.
\entry*[Software Development\hfill]
    Experience with the full software development lifecycle, including requirements gathering, design, implementation, testing, and maintenance. Agile and Scrum methodologies.
\entry*[Web Development]
    Skilled in both front-end and back-end development. Comfortable with React, Next.js, Tailwind, Angular, Node.js, and RESTful API integration.
\entry*[DevOps\hfill]
    Practical experience with CI/CD pipelines using Jenkins, Github Workflows, Docker, Kubernetes, AWS, and Azure cloud services. Also have experience deploying and maintaining machine learning models (MLOps).
\entry*[Machine Learning]
    Proficient with the entire machine learning pipeline, from dataset development to model deployment. Experience with PyTorch, TensorFlow, and Scikit-learn.
\entry*[Data Analysis]
    Proficient in data manipulation and analysis with Python (Pandas, NumPy, Matplotlib) and R.
\end{rubric}
\begin{rubric}{Experience}
%
\entry*[2021 -- present]%
\textbf{Research Scientist, SimulaMet} \par
Conducted interdisciplinary research in the application of AI in medicine and education, with a focus on algorithmic transparency.
Worked on projects to integrate AI technologies in practical healthcare solutions, enhancing patient outcomes and system efficiencies.
%
\entry*[2023 -- present]%
\textbf{CTO, Innsikt.AI} \par
Directing the technological strategy and innovation at a leading startup dedicated to the development of virtual child avatars for sophisticated police training, aimed at sensitive interviews with abused children.
Innovating in the application of AI and VR technologies to establish realistic and ethical training environments.
%
\entry*[2023 -- present]%
\textbf{Associated Professor, OsloMet} \par
Focusing on machine learning solutions for musculoskeletal disorders.
Responsible for student supervision, development of novel machine learning algorithms, and conducting influential academic research.
%
\entry*[2022 -- 2023]%
\textbf{Data Scientist, ForzaSys} \par
Designed and implemented algorithms for automated clipping and highlight generation in soccer matches.
Developed software for detecting and preventing match-fixing, enhancing the integrity of sports analytics.
%
\entry*[2017 -- 2018]%
\textbf{Front-End Developer, DHIS2} \par
Optimized table rendering processes and contributed significantly to developing applications for efficient data access and management.
%
\entry*[2014 -- 2016]%
\textbf{Full-Stack Developer, Axios AS} \par
Involved in the end-to-end development and design of loan management software for a US-based mortgage institution, ensuring compliance with industry standards.
%
\end{rubric}
\begin{rubric}{Selected Projects}
%
\entry*[2023]%
\textbf{EndoNet, American Society for Gastrointestinal Endoscopy} \par
Engineered a robust system for the acquisition and annotation of colonoscopy data from multiple healthcare centers across the United States. The platform aims to establish a comprehensive dataset to serve as a benchmark for future AI solutions in the field of gastrointestinal care.
%
\entry*[2023]%
\textbf{VALIDATE, European Union} \par 
Contributed to Trustworthy AI and Clinical Model Development within the VALIDATE project.
Focused on ensuring the reliability and ethical application of AI in clinical settings.
%
\entry*[2022]%
\textbf{Match-Fixing, Unibet and SEF} \par 
Collaborated with SEF and Unibet in developing a Next.js-based application to detect match-fixing in soccer.
Implemented data visualization and statistical analysis tools to identify potential fraud.
%
\entry*[2020]%
\textbf{Smittestopp, Simula Research Laboratory} \par
Key member of the data science team for Norway's COVID-19 contact tracing app, Smittestopp.
Responsible for visualizing user trajectories, validating contact events, and classifying transport modes, using Python.
%
\entry*[2019]%
\textbf{Gastrointestinal AI, Augere} \par
Partnered with Augere AS in creating datasets and AI models for the detection of gastrointestinal diseases.
Implemented models in Python using TensorFlow and PyTorch, enhancing diagnostic accuracy and efficiency.
%
\entry*[2018]%
\textbf{Fish Feeding, Spillfree Analytics} \par
Developed deep learning algorithms for fish detection and classification to automate feeding in aquaculture.
%
\entry*[2017]%
\textbf{Data Store Application, DHIS2} \par
Created a JavaScript-based web application for data management within the DHIS2 ecosystem at the University of Oslo.
The application was successfully integrated into the official DHIS2 platform, leading to part-time employment.
%
\entry*[2017]%
\textbf{Subscription Application, DNB} \par
Created a financial analytics application that scrutinizes nettbank transactions to provide users with an organized overview of daily, weekly, and yearly recurring expenses. The app also includes social features to share subscriptions with friends via Facebook. The app was developed using a React and Redux stack. There was no back-end, only DNB's internal APIs.
%
\end{rubric}
\begin{rubric}{Activities}
\entry*[2023 -- present]%
\textbf{Member of the Educational Council, NORA} \par 
Contributed to shaping the educational strategies and initiatives in AI across Norway as part of NORA's educational council. Played a key role in driving forward the agenda of integrating AI into broader educational frameworks nationally.
% 
\entry*[2018 -- present]%
\textbf{Challenge Organizer, Various International Workshops} \par
Organized and led multiple challenges across renowned workshops such as ACM MultiMedia, ICPR, ImageCLEF, DLAMC, ICMR, and Nordic AI Meet. Developed challenge themes, managed participant outreach, submission evaluation, and ensured smooth operation of challenge sessions.
%
\entry*[2018 -- present]%
\textbf{Main Organizing Committee Member, MediaEval} \par
Actively involved in the central organization of MediaEval, an annual international multimedia workshop. Responsible for overseeing key aspects of the workshop, including program planning, speaker coordination, and participant engagement.
% 
\entry*[2018 -- present]%
\textbf{Technical Program Committee Member, Various International Workshops} \par
Actively involved in the technical program committees for several high-profile conferences, including: ACM MM, ACM MM Asia, CBMS, ISMICT, MultiMediaEval, ICME, Nature Scientific Reports, and Nature Scientific Data.
%
\entry*[2022]%
\textbf{Conference Organizer, Norwegian Artificial Intelligence Society (NAIS)} \par
Contributed to organizing NAIS's annual conference, focusing on advancing and showcasing the latest developments in artificial intelligence in Norway.
%
\entry*[2019 -- 2023]%
\textbf{Editor, SIGMM Records} \par
Served as an editor for SIGMM Records, responsible for the interview section.
%
\entry*[2019 -- 2020]%
\textbf{Journal Club Coordinator, OsloMet AI Club} \par
Initiated and coordinated a bi-weekly journal club for the AI Lab at OsloMet. Facilitated discussions on recent advancements and research in AI, promoting a culture of continuous learning and academic discourse.
\end{rubric}
\begin{rubric}{Teaching}
\entry*[2023]%
\textbf{Master Course Instructor, Big Data Curation, Pipelines, and Management (GRA 4157), BI Norwegian Business School} \par
Led and instructed the undergraduate course "TK1101 – Digital Technology."
Focused on imparting foundational knowledge in digital technologies to technology majors.
Developed and delivered lectures, coursework, and assessments, guiding students through practical and theoretical aspects of the subject.
%
\entry*[2020]%
\textbf{Bachelor Course Instructor, Digital Technology (TK1101), Kristiania University College} \par
Led and instructed the undergraduate course "TK1101 – Digital Technology."
Focused on imparting foundational knowledge in digital technologies to technology majors.
Developed and delivered lectures, coursework, and assessments, guiding students through practical and theoretical aspects of the subject.
%
\entry*[2019]%
\textbf{Examination Assessor, Data Science (INS301), Kristiania University College} \par
Served as an examination assessor for the "INS301 – Data Science" course.
Responsible for evaluating student submissions, ensuring academic integrity, and providing fair and constructive feedback.
Contributed to the continuous improvement of examination standards and practices.
%
\entry*[2019]%
\textbf{Bachelor Course Instructor, Machine Learning (TEK300), Kristiania University College} \par
Conducted the undergraduate course "TEK300 – Machine Learning."
Covered key concepts and applications of machine learning, emphasizing practical skills and theoretical understanding.
%
\entry*[2019]%
\textbf{Master Course Instructor, Emerging Technologies (MS340), Kristiania University College} \par
Taught the master-level course "MS340 – Emerging Technologies."
Focused on advanced topics in applied computer science, including recent technological advancements and their applications.
%
\entry*[2019]%
\textbf{Examination Assessor, Introduction to Operating Systems (INF1060), University of Oslo} \par
Served as an examination assessor for "INF1060 – Introduction to Operating Systems."
Assessed student knowledge and understanding of operating systems, ensuring accurate and fair evaluation of their competencies.
\end{rubric}
\begin{rubric}{Education}
\entry*[2018 -- 2022]%
\textbf{Ph.D.~Computer Science, Oslo Metropolitan University} \par
Focused on developing methodologies to improve transparency and interpretability in AI systems used in healthcare, contributing to the field of ethical AI.
Collaborated with medical professionals to align technological advancements with practical healthcare needs.
%
\entry*[2016 -- 2018]%
\textbf{M.Sc.~Computer Science, University of Oslo} \par
Developed a system for automatic report generation of endoscopy procedures, using deep learning and explainable AI to explain the results to medical doctors.
% 
\entry*[2012 -- 2015]%
\textbf{B.Sc.~Information Technology, University of Agder} \par
Developed a cross-platform mobile application for the Norwegian Seamen's Church using Ionic Framework. This project demonstrated practical application of software development skills and provided a valuable service to the community, enhancing the accessibility and functionality of the Church's digital presence.
\end{rubric}

\end{document}