\begin{rubric}{Utvalgte Prosjekter}
%
\entry*[2023]%
\textbf{EndoNet, American Society for Gastrointestinal Endoscopy} Utviklet et robust system for innhenting og annotering av koloskopi-data fra flere helsesentre i USA. Plattformen har som mål å etablere en omfattende datasett som kan tjene som en referanse for fremtidige AI-løsninger innen tarmsykdom.
%
\entry*[2022]%
\textbf{Kampfiksing, Unibet og SEF} Samarbeidet med SEF og Unibet i design og implementering av en Next.js-basert applikasjon rettet mot å oppdage kampfiksing i fotball. Programvaren har avanserte statistikk- og datavisualiseringsmoduler for å identifisere potensielle svindelaktiviteter.
%
\entry*[2020]%
\textbf{Smittestopp, Simula} Bidro til den første versjonen av Smittestopp-applikasjonen som et medlem av data science-teamet. Ansvar inkluderte visualisering av brukerbaner, validering av kontakthendelser og utforming av algoritmer for klassifisering av ulike transportmetoder.
%
\entry*[2019]%
\textbf{Gastrointestinal AI, Augere AS} Samarbeidet med Augere AS for å konstruere datasett og AI-løsninger som letter automatisk påvisning av gastrointestinale sykdommer, og dermed effektivisere diagnostiske prosesser.
%
\entry*[2018]%
\textbf{Fiskemating, Spillfree Analytics} Designet og implementerte algoritmer for fiskedeteksjon og klassifisering med mål om å automatisere mating prosedyrer for fisk.
%
\entry*[2017]%
\textbf{Data Store App, DHIS2} Utviklet en webapplikasjon for å håndtere datalagring innenfor DHIS2-økosystemet ved Universitetet i Oslo. Applikasjonen ble deretter integrert i den offisielle DHIS2-plattformen, noe som førte til deltidsansettelse.
%
\entry*[2017]%
\textbf{Abonnement Oversikt App, DNB} Skapte en finansanalyseapplikasjon som gransker nettbanktransaksjoner for å gi brukerne en organisert oversikt over daglige, ukentlige og årlige utgifter. Appen inkluderer også sosiale funksjoner for å dele abonnementer med venner via Facebook.
%
\end{rubric}