\begin{rubric}{Selected Projects}
%
\entry*[2023]%
\textbf{EndoNet, American Society for Gastrointestinal Endoscopy} \par
Engineered a robust system for the acquisition and annotation of colonoscopy data from multiple healthcare centers across the United States. The platform aims to establish a comprehensive dataset to serve as a benchmark for future AI solutions in the field of gastrointestinal care.
%
\entry*[2023]%
\textbf{VALIDATE, European Union} \par 
Contributed to Trustworthy AI and Clinical Model Development within the VALIDATE project.
Focused on ensuring the reliability and ethical application of AI in clinical settings.
%
\entry*[2022]%
\textbf{Match-Fixing, Unibet and SEF} \par 
Collaborated with SEF and Unibet in developing a Next.js-based application to detect match-fixing in soccer.
Implemented sophisticated data visualization and statistical analysis tools to identify potential fraud.
%
\entry*[2020]%
\textbf{Smittestopp, Simula Research Laboratory} \par
Key member of the data science team for Norway's COVID-19 contact tracing app, Smittestopp.
Responsible for visualizing user trajectories, validating contact events, and classifying transport modes, using Python.
%
\entry*[2019]%
\textbf{Gastrointestinal AI, Augere} \par
Partnered with Augere AS in creating datasets and AI models for the detection of gastrointestinal diseases.
Implemented models in Python using TensorFlow and PyTorch, enhancing diagnostic accuracy and efficiency.
%
\entry*[2018]%
\textbf{Fish Feeding, Spillfree Analytics} \par
Developed deep learning algorithms for fish detection and classification to automate feeding in aquaculture.
Utilized Python with TensorFlow and PyTorch to create effective and reliable feeding systems.
%
\entry*[2017]%
\textbf{Data Store Application, DHIS2} \par
Created a JavaScript-based web application for data management within the DHIS2 ecosystem at the University of Oslo.
The application was successfully integrated into the official DHIS2 platform, leading to part-time employment.
%
\entry*[2017]%
\textbf{Subscription Application, DNB} \par
Created a financial analytics application that scrutinizes nettbank transactions to provide users with an organized overview of daily, weekly, and yearly recurring expenses. The app also includes social features to share subscriptions with friends via Facebook. The app was developed using a React and Redux stack. There was no back-end, only DNB's internal APIs.
%
\end{rubric}